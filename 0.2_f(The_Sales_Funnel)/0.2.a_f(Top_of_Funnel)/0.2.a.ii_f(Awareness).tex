0.2.a.ii_f(Awareness) 

# Awareness as a Function: Complete Mathematical and Theoretical Foundations

**Awareness operates as the gating function between stimulus and response across all human decision-making.** This comprehensive mapping reveals that marketing science and cognitive science converged on remarkably similar mathematical insights: awareness is scarce, capacity-limited, probabilistically distributed, and subject to threshold effects. From E. St. Elmo Lewis's 1898 sales funnel to Karl Friston's 2010 free energy principle, the core problem remains unchanged—how do signals break through noise to enter conscious processing? The frameworks documented here represent the complete intellectual lineage informing modern attention measurement, from GRP formulas used in media buying to d-prime equations in psychophysics laboratories.

---

## Part 1: Marketing and Brand Awareness Frameworks

### The hierarchy of effects tradition (1898–1980)

The foundational idea that awareness precedes action emerged from direct-mail advertising in the late 19th century. **E. St. Elmo Lewis** (1872–1948) first articulated the principle in an 1898 *Printers' Ink* article: "attract attention, maintain interest, create desire." The fourth element—"get action"—appeared later. Lewis drew explicitly on William James's psychology of attention, making this the first documented bridge between cognitive science and advertising theory. The **AIDA acronym** itself was coined by C.P. Russell in 1921, though Lewis's conceptual framework preceded it by over two decades.

**Robert Lavidge and Gary Steiner** formalized the concept in their landmark 1961 paper "A Model for Predictive Measurements of Advertising Effectiveness" (*Journal of Marketing*). Their six-stage hierarchy—Awareness → Knowledge → Liking → Preference → Conviction → Purchase—explicitly mapped onto three psychological domains: cognitive (thinking), affective (feeling), and conative (doing). This tripartite structure influenced all subsequent models. The same year, **Russell Colley's DAGMAR** framework (*Defining Advertising Goals for Measured Advertising Results*, ANA, 1961) introduced the revolutionary principle that advertising should be measured by communication effects, not sales. His four-stage model—Awareness → Comprehension → Conviction → Action—spawned modern brand tracking methodology.

**Richard Vaughn's FCB Grid** (1980, *Journal of Advertising Research*) integrated these hierarchies with involvement theory, creating a 2×2 matrix crossing High/Low Involvement with Thinking/Feeling. The key insight: the Think-Feel-Do sequence reorders based on product category. High-involvement/thinking products (cars, insurance) follow Learn→Feel→Do; low-involvement/feeling products (beer, snacks) follow Do→Feel→Learn. This remains the standard framework in advertising planning.

| Model | Author(s) | Year | Sequence | Key Innovation |
|-------|-----------|------|----------|----------------|
| AIDA | E. St. Elmo Lewis | 1898 | Attention→Interest→Desire→Action | First hierarchical model |
| Hierarchy of Effects | Lavidge & Steiner | 1961 | 6 stages mapped to cognition/affect/conation | Psychological domains |
| DAGMAR | Russell Colley | 1961 | Awareness→Comprehension→Conviction→Action | Measurable objectives |
| FCB Grid | Richard Vaughn | 1980 | Variable by quadrant | Involvement × processing mode |

**Critiques**: Vakratsas and Ambler's 1999 meta-analysis of 250+ papers found little empirical support for strict sequential processing. Modern neuroscience confirms that cognitive and affective processing occur simultaneously, not sequentially.

---

### Media mathematics: Reach, frequency, and carryover effects

The quantitative infrastructure of media planning rests on equations developed between 1950 and 1980 that remain industry standard today.

**Gross Rating Points (GRP)** emerged organically in 1950s television advertising:
```
GRP = Reach (%) × Average Frequency
```
One GRP represents reaching 1% of the target population once. The formula's elegance masks its limitation: it treats all exposures as equivalent regardless of attention quality.

**Herbert Krugman's Three-Exposure Theory** (1972, "Why Three Exposures May Be Enough," *Journal of Advertising Research*) proposed that advertising has only three psychological states: (1) "What is it?" (curiosity), (2) "What of it?" (evaluation), and (3) reminder/decision. Krugman argued there is "no such thing as a fourth exposure psychologically—fours, fives, etc., are repeats of the third." **Michael Naples** codified this as the "3+ frequency" standard in his 1979 ANA report, which dominated media planning for two decades.

**Erwin Ephron's Recency Theory** (1995, "More Weeks, Less Weight," *Journal of Advertising Research*) challenged effective frequency orthodoxy. Ephron argued that a single exposure within the purchase decision window matters more than accumulated frequency. His recommendation—maximize weekly reach rather than building frequency—transformed media planning for FMCG brands. John Philip Jones's single-source research confirmed that one exposure in the seven days before purchase has far greater impact than multiple earlier exposures.

**Simon Broadbent's Adstock Model** (1979, *Journal of the Market Research Society*) provided the mathematical framework for modeling awareness decay:
```
Adstock_t = Advertising_t + λ × Adstock_(t-1)
```
Where λ (0 < λ < 1) is the decay parameter. The **half-life** calculation:
```
Half-life = ln(0.5) / ln(λ)
```
Industry benchmarks: FMCG brands average **2.5 weeks** half-life; academic studies suggest **7–12 weeks** for brand awareness effects. This equation underlies all modern marketing mix models.

**Advertising response functions** follow three patterns:
- **Linear**: Sales = α + β × Advertising (rarely observed)
- **Concave/Diminishing Returns**: Sales = α × Advertising^β where 0 < β < 1 (most common empirically)
- **S-Curve/Logistic**: Threshold effect before response, then saturation
```
Response = Max × (Spend^α) / (K^α + Spend^α)    [Hill Function]
```

The **Vidale-Wolfe Model** (1957, *Operations Research*) introduced dynamics:
```
dS/dt = r × A × (M - S)/M - δ × S
```
Where S = sales rate, A = advertising spend, M = market saturation, r = response constant, δ = decay rate. This was the first differential equation model of advertising effects.

---

### Share of Voice and the Ehrenberg-Bass revolution

**John Philip Jones** (1990, *Harvard Business Review*) documented the empirical relationship between Share of Voice (SOV) and Share of Market (SOM):
```
Extra Share of Voice (ESOV) = SOV - SOM
```
Challenger brands require positive ESOV to grow; market leaders can profit with negative ESOV. **Les Binet and Peter Field's** IPA research suggests:
```
Expected Annual Market Share Growth ≈ 0.5% per 10% ESOV
```

The **Ehrenberg-Bass Institute** (founded by Andrew Ehrenberg) transformed brand science through rigorous empirical analysis. Their **NBD-Dirichlet Model** (Goodhardt, Ehrenberg & Chatfield, 1984, *Journal of the Royal Statistical Society*) describes purchase behavior with three parameters: M (mean category purchase rate), K (purchase frequency diversity), and S (brand propensity diversity). Key equations:

**Brand Penetration**:
```
b = 1 - Σ(n=0→∞) P_n × p(0|n)
```

**Double Jeopardy Law**: Smaller brands suffer twice—fewer buyers AND lower loyalty. Expression: w(1-b) = constant, where w = purchase frequency and b = brand penetration.

**Duplication of Purchase Law**:
```
b_XY = D × b_X
```
Brand Y's buyers who also buy brand X equals the Duplication coefficient times brand X's penetration. This implies no systematic brand segmentation in most categories.

**Byron Sharp's "How Brands Grow"** (2010) synthesized Ehrenberg-Bass findings into marketing's most influential modern text. Key constructs:
- **Mental Availability**: Brand's propensity to be noticed or retrieved in buying situations
- **Physical Availability**: Ease of finding and buying the brand
- **Category Entry Points (CEPs)**: The cues buyers use to access memory (developed by **Jenni Romaniuk**)

Mental Market Share = Brand's CEP associations / Total category CEP associations

**Kevin Lane Keller's CBBE Pyramid** (1993, 2001, *Journal of Marketing*) provides an alternative framework. Brand Salience forms the foundation, defined by depth (how easily recalled) and breadth (range of situations triggering recall). The pyramid ascends through Performance/Imagery → Judgments/Feelings → Resonance.

---

### Attention economics: From Simon to digital metrics

**Herbert Simon** established the theoretical foundation in his 1971 lecture "Designing Organizations for an Information-Rich World":

> "A wealth of information creates a poverty of attention and a need to allocate that attention efficiently among the overabundance of information sources that might consume it."

Simon connected attention scarcity to his earlier work on **bounded rationality** (1955, *Quarterly Journal of Economics*): humans "satisfice" rather than optimize because cognitive capacity is limited. This remains the master concept underlying all attention economics.

**Michael Goldhaber** (1997, *Wired*) declared attention "the currency of the New Economy," arguing that unlike information, human attention is "truly individual and scarce." **Georg Franck** (1998, *Ökonomie der Aufmerksamkeit*) developed the concept of "mental capitalism"—attention as literal capital that earns interest. Celebrity functions as a "stock exchange of attention capital."

**Thomas Davenport and John Beck** (*The Attention Economy*, 2001) operationalized these ideas for business: "Understanding and managing attention is now the single most important determinant of business success."

Modern **digital attention measurement** has evolved beyond simple impressions. The **IAB/MRC Viewability Standards** (2014) established minimum thresholds:
- Display: ≥50% pixels visible for ≥1 second
- Video: ≥50% pixels visible for ≥2 seconds

But viewability ≠ attention. **Lumen Research** (Mike Follett, founded 2013) pioneered eye-tracking based metrics:
```
APM = (% of ads viewed × average viewing time) × 1000 impressions
aCPM = CPM / APM × 1,000
```
Their finding: only **35% of "viewable" ads are actually looked at**.

**Adelaide** (Marc Guldimann, 2019) developed the **AU (Attention Unit)** metric—a 0–100 score combining eye-tracking data, media quality signals, and outcome data to predict attention probability rather than duration. This addresses the "Attentive Audience Paradox" where older audiences dwell longer but don't necessarily engage more.

**Karen Nelson-Field** (Amplified Intelligence, *The Attention Economy and How Media Works*, 2020) established that **1.5 seconds of active attention** is sufficient for memory encoding. Her research shows attention metrics are 7× more effective at predicting brand awareness than viewability.

---

## Part 2: Cognitive Science and Attention Theory

### Signal detection and information theory foundations

**Signal Detection Theory (SDT)** separates perceptual sensitivity from response bias. **Green and Swets** (*Signal Detection Theory and Psychophysics*, 1966) formalized:
```
d' = z(Hit Rate) - z(False Alarm Rate)
```
Where z() is the inverse normal CDF. The d' (d-prime) measures discriminability independent of criterion setting. **Criterion measures**:
```
c = -0.5 × (z(HR) + z(FAR))    [criterion location]
β = exp((z(FA)² - z(H)²) / 2)    [likelihood ratio]
```
**ROC curves** plot Hit Rate against False Alarm Rate across criterion levels; Area Under Curve (AUC) quantifies overall discrimination ability.

**Claude Shannon's Information Theory** (1948, *Bell System Technical Journal*) established channel capacity:
```
C = B × log₂(1 + S/N)
```
Where C = capacity in bits/second, B = bandwidth, S/N = signal-to-noise ratio.

**George Miller** (1956, *Psychological Review*) applied information theory to cognition, discovering that immediate memory span is limited to **7 ± 2 chunks**—independent of information bits per chunk. Miller emphasized **chunking** as the mechanism for expanding effective capacity. **Nelson Cowan** (2001, *Behavioral and Brain Sciences*) revised this to **4 ± 1** items when rehearsal is prevented:
```
K = (hit rate + correct rejection rate - 1) × N
```

---

### Selective attention: Filter models and their evolution

**Donald Broadbent's Filter Model** (1958, *Perception and Communication*) proposed that a selective filter operates on physical characteristics (location, pitch, voice) early in processing, blocking unattended information before semantic analysis:
```
Stimuli → Sensory Buffer → FILTER → Limited-capacity Processor → Response
```

**Anne Treisman's Attenuation Theory** (1964, *British Medical Bulletin*) modified this: instead of blocking, the filter merely "turns down the volume." Important stimuli (one's name, danger words) have permanently low thresholds and break through:
```
If (Attenuated Signal) ≥ Threshold → Conscious Awareness
```
Threshold varies by: subjective importance, recent activation (priming), contextual expectancy, and biological significance.

**Deutsch and Deutsch's Late Selection Model** (1963, *Psychological Review*) proposed all stimuli receive full semantic processing; selection occurs later based on importance/pertinence. Evidence: conditioned words produce galvanic skin responses even in unattended channels (Moray, 1969).

**Daniel Kahneman's Capacity Model** (1973, *Attention and Effort*) shifted focus from filtering to resource allocation. Attention is a limited, undifferentiated pool of mental "fuel":
```
Total_Capacity = g(Arousal)    [inverted-U function]
```
Performance = f(Resources_allocated, Task_demands). When demands exceed capacity, performance suffers.

**Christopher Wickens' Multiple Resource Theory** (1980, 2002) identified four dimensions of resource separation:
- **Stages**: Perceptual/Cognitive vs. Response
- **Modalities**: Auditory vs. Visual
- **Codes**: Spatial vs. Verbal
- **Visual Channels**: Focal vs. Ambient

Prediction: interference increases with shared resources across dimensions. This explains why auditory-verbal + visual-spatial tasks combine well (different tanks), while visual-verbal + visual-spatial compete (same tank).

---

### Computational models of visual attention

**Anne Treisman's Feature Integration Theory** (Treisman & Gelade, 1980, *Cognitive Psychology*) distinguished:
- **Pre-attentive processing**: Features (color, orientation) processed in parallel; targets "pop out"
- **Focused attention**: Required to bind features together (solving the "binding problem")

Search slope equations:
```
Feature search (parallel): RT = a + b (flat, ~0 ms/item slope)
Conjunction search (serial): RT_positive = a + b × N/2; RT_negative = a + b × N
```

**The Itti-Koch Saliency Model** (1998, *IEEE PAMI*) operationalized bottom-up attention computationally:

1. Input decomposed into 9 spatial scales via Gaussian pyramids
2. Feature channels extracted: Intensity, Color (RG, BY opponency), Orientation (4 angles)
3. Center-surround operations create feature maps (42 total)
4. **Normalization operator N(·)** promotes maps with few strong peaks
5. Conspicuity maps combined: **S = (1/3)[N(Ī) + N(C̄) + N(Ō)]**
6. Winner-take-all network selects most salient location

The model predicts first **~250ms** of viewing (pre-attentive processing). Software: SaliencyToolbox (MATLAB), pySaliencyMap (Python).

**Jeremy Wolfe's Guided Search** (1989–2021) bridges bottom-up saliency with top-down guidance:
```
A_i = w_BU × BU_i + Σ_d(w_d × TD_d,i) + noise
```
**Guided Search 6.0** (2021, *Psychonomic Bulletin & Review*) integrates five priority sources: bottom-up salience, top-down guidance, history/priming, reward/value, and scene semantics. Selection rate: **~20 Hz** attention deployment.

**Deep learning models** now outperform classical saliency. **DeepGaze IIE** (Linardos et al., 2021) achieves ~93% of human inter-observer consistency using pretrained vision transformer features without task-specific training.

---

### Working memory, consciousness, and predictive processing

**Baddeley's Working Memory Model** (1974, updated 2000) comprises:
- **Central Executive**: Attentional control, coordinates subsystems
- **Phonological Loop**: Verbal/auditory information (~2 seconds rehearsal capacity)
- **Visuospatial Sketchpad**: Visual/spatial information
- **Episodic Buffer** (2000): Integrates information across subsystems and long-term memory

**Cognitive Load Theory** (John Sweller, 1988, *Cognitive Science*) distinguishes:
```
Total Load = Intrinsic Load + Extraneous Load
```
Overload occurs when Total Load > Working Memory Capacity. Implications for advertising: minimize extraneous load through design optimization.

**Global Workspace Theory** (Bernard Baars, 1988) uses a theater metaphor: consciousness is the "spotlight" on a limited "stage," broadcasting selected information to a vast "audience" of unconscious processors. **Dehaene and Changeux** (1998, 2003) provided neural implementation: long-range workspace neurons in prefrontal-parietal cortex show "ignition"—sudden, late (~300ms), sustained firing when threshold crossed. **Signatures of consciousness**: P3b wave, gamma synchrony, long-distance coherence.

**Predictive Processing** (Rao & Ballard, 1999, *Nature Neuroscience*) models the brain as a hierarchical prediction machine:
```
Prediction: ŷ_i = W_i · r_(i+1)
Prediction Error: e_i = y_i - ŷ_i
```
Feedforward connections carry errors; feedback carries predictions. Attention = precision-weighting of prediction errors.

**Karl Friston's Free Energy Principle** (2010, *Nature Reviews Neuroscience*) provides a unified framework:
```
F = E_q[ln q(s) - ln p(s,o)]
```
Equivalently: F = KL-divergence between beliefs and true posterior + surprisal. The brain minimizes free energy through: (1) perceptual inference (updating beliefs), (2) learning (updating model), (3) action (changing inputs). **Attention modulates precision**:
```
Precision (π) = 1/variance
```
High-precision prediction errors gain more weight in updating beliefs.

**Integrated Information Theory** (Giulio Tononi, 2004) proposes consciousness = integrated information (Φ):
```
Φ = min_partition D(p(X_t+1|X_t) || Π_i p(X^i_t+1|X^i_t))
```
The theory predicts the posterior cortical "hot zone" is more conscious than prefrontal cortex. Major critique: calculating Φ is computationally intractable (NP-hard) for systems larger than ~15 elements.

---

## Part 3: Cross-Domain Synthesis

### How cognitive theory shaped advertising science

The bridges between cognitive science and advertising are surprisingly well-documented:

1. **E. St. Elmo Lewis** explicitly cited William James's attention psychology in developing AIDA (1898)
2. **Herbert Krugman** based three-exposure theory on the mere-exposure effect and latent inhibition from cognitive psychology
3. **Information theory** (Shannon, Miller) informed message design: keep key points under 4–7 items
4. **Signal Detection Theory** provides framework for measuring ad noticeability independent of response bias
5. **Feature Integration Theory** explains why distinctive visual features (bright colors, motion) capture pre-attentive attention
6. **Kahneman's capacity model** explains why cognitively loaded consumers process ads superficially
7. **Multiple Resource Theory** informs multimodal ad design: combine audio-verbal with visual-spatial

### The attention funnel: Unified mathematical framework

Across domains, awareness functions as the first filter in a multi-stage process:

**Marketing Funnel**:
```
Exposure → Attention → Awareness → Processing → Attitude → Behavior
```

**Cognitive Funnel**:
```
Sensation → Feature Detection → Saliency Competition → Working Memory → Decision
```

**Shared mathematical properties**:
- **Capacity limits**: ~4 items in working memory; ~7 brands in consideration set
- **Decay functions**: Adstock λ ≈ 0.75 (awareness); trace decay ~1–2 seconds (sensory memory)
- **Threshold effects**: Minimum exposure for awareness; ignition threshold for consciousness
- **Probabilistic retrieval**: Brand salience = CEP-based probability; memory retrieval = activation-based probability

### Integration of metrics across paradigms

| Marketing Metric | Cognitive Equivalent | Mathematical Relationship |
|-----------------|---------------------|--------------------------|
| Brand Awareness (%) | Recognition probability | p(recall) = f(activation strength, threshold) |
| Effective Frequency (3+) | Memory consolidation | Repetition strengthens trace |
| Adstock decay (λ) | Memory decay | Exponential forgetting curve |
| Mental Availability | Spreading activation | Network accessibility |
| Attention time (seconds) | Fixation duration | Encoding time = f(frequency, eccentricity) |
| d' (sensitivity) | Ad discriminability | Signal detection independent of bias |

---

## Master Reference: Key Thinkers and Equations

### Marketing and Advertising

| Framework | Author(s) | Year | Key Equation/Model |
|-----------|-----------|------|--------------------|
| AIDA | E. St. Elmo Lewis | 1898 | Attention→Interest→Desire→Action |
| Hierarchy of Effects | Lavidge & Steiner | 1961 | 6 stages: Awareness→Knowledge→Liking→Preference→Conviction→Purchase |
| DAGMAR | Russell Colley | 1961 | Awareness→Comprehension→Conviction→Action |
| FCB Grid | Richard Vaughn | 1980 | 2×2: Involvement × Thinking/Feeling |
| GRP | Industry standard | 1950s | GRP = Reach × Frequency |
| Three Exposures | Herbert Krugman | 1972 | Psychological threshold model |
| Effective Frequency | Michael Naples | 1979 | 3+ exposures optimal |
| Recency Theory | Erwin Ephron | 1995 | One exposure in purchase window |
| Adstock | Simon Broadbent | 1979 | A_t = T_t + λ × A_(t-1) |
| Vidale-Wolfe | Vidale & Wolfe | 1957 | dS/dt = r×A×(M-S)/M - δ×S |
| SOV-SOM | John Philip Jones | 1990 | ESOV = SOV - SOM |
| NBD-Dirichlet | Ehrenberg et al. | 1984 | Brand choice probability model |
| Mental Availability | Byron Sharp/Romaniuk | 2004–2010 | CEP-based salience |
| CBBE Pyramid | Kevin Lane Keller | 1993 | Salience→Meaning→Response→Resonance |
| AU Metric | Adelaide/Guldimann | 2019 | 0–100 attention probability score |

### Cognitive Science

| Framework | Author(s) | Year | Key Equation/Model |
|-----------|-----------|------|--------------------|
| Channel Capacity | Claude Shannon | 1948 | C = B × log₂(1 + S/N) |
| 7 ± 2 Chunks | George Miller | 1956 | Immediate memory span |
| Filter Model | Donald Broadbent | 1958 | Early physical selection |
| Attenuation | Anne Treisman | 1964 | Signal reduction + threshold |
| Late Selection | Deutsch & Deutsch | 1963 | Post-semantic selection |
| Signal Detection | Green & Swets | 1966 | d' = z(HR) - z(FAR) |
| Capacity Model | Daniel Kahneman | 1973 | Flexible resource pool |
| Feature Integration | Treisman & Gelade | 1980 | Parallel features + serial binding |
| Multiple Resources | Christopher Wickens | 1980 | 4-D interference model |
| Global Workspace | Bernard Baars | 1988 | Broadcast to unconscious processors |
| Cognitive Load | John Sweller | 1988 | Total = Intrinsic + Extraneous |
| Itti-Koch Saliency | Itti & Koch | 1998 | S = (1/3)[N(Ī) + N(C̄) + N(Ō)] |
| Predictive Coding | Rao & Ballard | 1999 | Prediction error minimization |
| 4 ± 1 Chunks | Nelson Cowan | 2001 | True capacity without rehearsal |
| Free Energy | Karl Friston | 2010 | F = E_q[ln q(s) - ln p(s,o)] |
| Guided Search 6.0 | Jeremy Wolfe | 2021 | 5-source priority map |

---

## Conclusion: The unified logic of awareness

The intellectual archaeology of awareness reveals a remarkable convergence: marketing practitioners and cognitive scientists independently discovered the same fundamental constraints. Herbert Simon's "poverty of attention" (1971) echoes through every subsequent framework—from Broadbent's bottleneck to Adelaide's AU metric. The mathematical formalizations differ in precision (d-prime equations vs. GRP calculations) but share underlying logic: awareness is a **competitive, capacity-limited, threshold-gated process** where signals must overcome noise to achieve conscious representation.

Three principles emerge across all frameworks:

**First, awareness is probabilistic, not deterministic.** Whether measured as brand recall rate, fixation probability, or d' sensitivity, awareness is always a probability distribution, not a binary state. The NBD-Dirichlet model, signal detection theory, and predictive processing all treat awareness as stochastic.

**Second, awareness has architectural constraints.** Miller's 7±2, Cowan's 4±1, Wickens' multiple resources, and working memory models all identify hard limits on simultaneous processing. These constraints explain both cognitive load effects in advertising and the effectiveness of chunking in brand design.

**Third, awareness involves active prediction, not passive reception.** From Treisman's thresholds to Friston's precision-weighting, modern theory emphasizes that prior expectations shape what enters awareness. This validates marketing practices around brand consistency, mental availability building, and contextual targeting.

The field continues evolving toward attention-based measurement (Adelaide, Lumen, Amplified Intelligence) and computational prediction (DeepGaze, saliency models). But the fundamental insight remains what Simon articulated fifty years ago: in an information-rich world, awareness is the scarce resource that determines all downstream effects. Every equation documented here—from GRP = Reach × Frequency to F = E_q[ln q(s) - ln p(s,o)]—represents an attempt to model that scarcity mathematically.
